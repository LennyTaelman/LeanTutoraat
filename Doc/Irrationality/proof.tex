\documentclass[11pt]{article}
\usepackage[utf8]{inputenc}
\usepackage{amsmath}
\usepackage{amssymb}
\usepackage{amsthm}
\usepackage{geometry}
\geometry{margin=1in}

\title{Proof Sketch: Irrationality of $e$}
\author{}
\date{}

\newtheorem{theorem}{Theorem}
\newtheorem{lemma}{Lemma}
\newtheorem{corollary}{Corollary}
\newtheorem{definition}{Definition}

\def\N{\mathbb{N}}
\def\Z{\mathbb{Z}}
\def\R{\mathbb{R}}

\begin{document}

\maketitle


\section{The sequence $a_n = 1/n!$}

\begin{lemma}[Key bound on the factorial function]\label{lemma:fac-bound}
For all $n > 0$ and $k \in \mathbb{N}$,
\[
(n+k)! \geq 2^k \cdot n!.
\]
\end{lemma}

\begin{proof}
The proof proceeds by induction on $k$. For the base case $k = 0$, we have $(n+0)! = 2^0 \cdot n!$.
For the inductive step, assume $(n+k)! \geq 2^k \cdot n!$. Then
\begin{align*}
(n+k+1)! &= (n+k+1) \cdot (n+k)! \\
&\geq (n+k+1) \cdot 2^k \cdot n! \\
&\geq 2 \cdot 2^k \cdot n! = 2^{k+1} \cdot n!,
\end{align*}
where the last inequality uses that $n+k+1 \geq 2$ when $n > 0$.
\end{proof}

Remark: the condition that $n>0$ cannot be dropped.

\begin{definition}
Set $a_n = 1/n!$. This defines a function $a\colon \N \to \R, n \mapsto a_n$.
\end{definition}


\begin{lemma}[Key bound on $a_n$]\label{lemma:a-bound}
For all $n \geq 1$ and $k \in \mathbb{N}$, we have
\[
a_{n+k}  \leq \left(\frac{1}{2}\right)^k \cdot a_n.
\]
\end{lemma}

\begin{proof}
Follows from Lemma~\ref{lemma:fac-bound}, using the fact that both sides in the
inequality in Lemma~\ref{lemma:fac-bound} are positive. 
\end{proof}

\section{The partial sums $s_n = a_0 + \cdots + a_{n-1}$}

Define the partial sums
\[
s_n = \sum_{i=0}^{n-1} a_i = \frac{1}{0!} + \frac{1}{1!} + \cdots + \frac{1}{(n-1)!} \in \R
\]
for $n \in \mathbb{N}$. Note that there are $n$ terms in the sum. We follow the usual
convention that $s_0 = 0$ (empty sum).

The key inequality for the partial sums is:

\begin{definition}
Consider the partial sum of the geometric series
\[
  g_k = 1 + \frac{1}{2} + \cdots + \left(\frac{1}{2}\right)^{k-1},
\]
where again $g_0 = 0$ (empty sum).
\end{definition}

\begin{lemma}\label{lemma:g-bound}$g_k < 2$.
\end{lemma}
 
\begin{proof}
Indeed, we have $g_k = 2 - (1/2)^{k-1}$.
\end{proof}

\begin{lemma}\label{lemma:s-below-geometric}
For all $m$ and all $n \geq 1$ and $k \in \mathbb{N}$, we have
\[
s_{n+k} \leq s_n + a_n \cdot g_k.
\]
\end{lemma}

\begin{proof}
The proof proceeds by induction on $k$. For $k = 0$, we have $s_{n+0} = s_n + a_n \cdot g_0$.

For the inductive step, assume $s_{n+k} \leq s_n + a_n \cdot g_k$. Then
\begin{align*}
s_{n+k+1} &= s_{n+k} + a_{n+k} \\
&\leq s_n + a_n \cdot g_k + a_{n+k} \\
&\leq s_n + a_n \cdot g_k + \left(\frac{1}{2}\right)^k \cdot a_n \\
&= s_n + a_n \cdot \left(g_k + \left(\frac{1}{2}\right)^k\right) \\
&= s_n + a_n \cdot g_{k+1},
\end{align*}
where we used the bound $a_{n+k} \leq (1/2)^k \cdot a_n$ from Lemma~\ref{lemma:a-bound}.
\end{proof}

\begin{lemma}\label{lemma:s-bound}
For all $n \geq 1$ and $k \in \mathbb{N}$,
\[
s_{n+k} < s_n + 2 \cdot a_n.
\]
\end{lemma}

\begin{proof}
Follows from Lemma~\ref{lemma:s-below-geometric} and Lemma~\ref{lemma:g-bound}.
\end{proof}

It will be convenient to establish the following variant:

\begin{lemma}\label{lemma:s-bound-total}
For all $n \geq 1$ and all $m \geq 0$ we have
\[
  s_m < s_n + 2 \cdot a_n
\]
\end{lemma}

\begin{proof}
Follows by case distinction: if $m>n$ then this is Lemma~\ref{lemma:s-bound}, and if $m\leq n$
then this follows from $s_m \leq s_n$ and $a_n > 0$.
\end{proof}


\section{The number $e = \lim_{n \to \infty} s_n$}

The number $e$ is defined as the limit of the sequence $(s_n)$:
\[
e = \lim_{n \to \infty} s_n = \sum_{n=0}^{\infty} \frac{1}{n!}.
\]

Since $(s_n)$ is a strictly increasing sequence (as $a_n > 0$ for all $n$) and bounded
above (by Lemma~\ref{lemma:s-bound-total}), the limit exists. 

\begin{lemma}[Characterisation of $e$]\label{lemma:e-characterisation} The number $e$ satisfies:
    \begin{enumerate}
       \item For all $n$, we have $s_n < e$;
       \item If $c\in \R$ and $s_n \leq c$ for all $n$, then $e \leq c$.
    \end{enumerate}
Moreover, $e$ is the \emph{only}  real number that satisfies these two properties. \qed
\end{lemma}

Combining the second point with Lemma~\ref{lemma:s-bound-total} we obtain:

\begin{lemma}[Key bound on $e$]\label{lemma:e-bound}
  For all $n>0$ we have $e \leq s_n + 2 \cdot a_n$
\end{lemma}


\section{Integrality of $n! \cdot s_{n+1}$}

\begin{definition}We say that a real number $x$ is \emph{integral} if $x
\in \Z$. (In Lean: if there exists an integer $N$ such that $x = \uparrow N$.)
\end{definition}

\begin{lemma}\label{lemma:fac-mul-a-integral}
  For all $m \leq n$ we have $n! \cdot a_m$ is an integer.
\end{lemma}

\begin{proof}By induction on $n$ starting at $n=m$. The base step is the
statement that $n! \cdot a_n = 1$ is an integer. For the inductive step, assume
that $n! \cdot a_m$ is an integer. Then
\[
   (n+1)! \cdot a_m = (n+1) \cdot n! \cdot a_m = (n+1) \cdot (\text{some integer}),
\]
hence $(n+1)! \cdot a_m$ is an integer.
\end{proof}

\begin{lemma}For all $n$ we have that $n! \cdot s_{n+1}$ is an integer.
\end{lemma}

\begin{proof}
Induction on $n$. For $n=0$ we have $0! \cdot s_1 = 1$, which is an integer. For
the inductive step, assume that $n! \cdot s_{n+1}$ is an integer. Then we have
\[
  (n+1)! \cdot s_{n+2} = (n+1)! \cdot (s_{n+1} + a_{n+1}) = (n+1) \cdot (n! \cdot s_{n+1}) + (n+1)! \cdot a_{n+1}.
\]
The first term is an integer by the induction hypothesis, and the second term is an integer by Lemma~\ref{lemma:fac-mul-a-integral}.
\end{proof}

\section{The tail $t_n = e - s_n$}

Consider the `tail' of the series:
\[
t_n := e - s_n = \frac{1}{n!} + \frac{1}{(n+1)!} + \cdots.
\]

\begin{lemma}\label{lemma:t-positive}$t_n > 0$ for all $n$.
\end{lemma}

\begin{proof} Follows from the inequality $s_n < e$ of Lemma~\ref{lemma:e-characterisation}.
\end{proof}

\begin{lemma}\label{lemma:t-bound}
If $n > TODO$ then $t_n \leq 2 \cdot a_n$
\end{lemma}

\begin{proof}
Follows from Lemma~\ref{lemma:t-bound} and Lemma~\ref{lemma:a-bound}.
\end{proof}

\begin{lemma}\label{lemma:fac-mul-t-bound}
For all $n \geq 2$,
\[
0 < n! \cdot t_{n+1} < 1.
\]
\end{lemma}

\begin{proof}
The positivity follows from $t_{n+1} > 0$. For the upper bound, we use:
\begin{align*}
n! \cdot t_{n+1} &\leq n! \cdot 2 \cdot a_{n+1} \\
&= n! \cdot 2 \cdot \frac{1}{(n+1)!} \\
&= \frac{2}{n+1} \\
&\leq \frac{2}{3} < 1,
\end{align*}
where the last inequality uses $n \geq 2$.
\end{proof}


\begin{lemma}
If $n\geq 2$, then $n! \cdot t_{n+1}$ is not integral.
\end{lemma}

\begin{proof}
Direct consequence of Lemma~\ref{lemma:fac-mul-t-bound}: there is no integer strictly
between $0$ and $1$.
\end{proof}

\section{Irrationality of $e$}


\begin{proof}[Proof sketch]
Suppose for contradiction that $n! \cdot e = N$ for some integer $N$. Since $n! \cdot s_{n+1}$ is an integer (say $M$), we have
\[
n! \cdot t_{n+1} = n! \cdot (e - s_{n+1}) = N - M \in \mathbb{Z}.
\]
But $0 < n! \cdot t_{n+1} < 1$, and there are no integers strictly between $0$ and $1$, which is a contradiction.
\end{proof}

\section{Proof that $e$ is not rational}

\begin{theorem}[Irrationality of $e$]
The number $e$ is irrational.
\end{theorem}

\begin{proof}[Proof sketch]
Suppose for contradiction that $e = p/q$ for some natural numbers $p, q$ with $q > 0$.

Then for $n = q + 1 \geq 2$, we have:
\[
(q+1)! \cdot e = (q+1)! \cdot \frac{p}{q} = (q+1) \cdot q! \cdot \frac{p}{q} = (q+1) \cdot (q-1)! \cdot p \in \mathbb{N}.
\]

But by the previous theorem, $(q+1)! \cdot e$ cannot be an integer, which is a contradiction.
\end{proof}

\end{document}

