\documentclass[11pt]{article}
\usepackage[utf8]{inputenc}
\usepackage{amsmath}
\usepackage{amssymb}
\usepackage{amsthm}
\usepackage{geometry}
\geometry{margin=1in}

\title{Proof Sketch: Irrationality of $e$}
\author{}
\date{}

\newtheorem{theorem}{Theorem}
\newtheorem{lemma}{Lemma}
\newtheorem{corollary}{Corollary}
\newtheorem{definition}{Definition}

\def\N{\mathbb{N}}
\def\Z{\mathbb{Z}}
\def\Q{\mathbb{Q}}
\def\R{\mathbb{R}}

\begin{document}

\maketitle


\section{Bound on the factorial function}

\begin{theorem}[{\tt fac\_bound}]\label{thm:fac-bound}
  For all $n > 0$ and $k \in \mathbb{N}$,
  \[
  (n+k)! \geq 2^k \cdot n!.
  \]
  \end{theorem}
  
\begin{proof}
Induction on $k$. For  $k = 0$, clear. Assume $(n+k)! \geq 2^k \cdot n!$. Then
    \begin{align*}
  (n+k+1)! &= (n+k+1) \cdot (n+k)! \\
  &\geq (n+k+1) \cdot 2^k \cdot n! \\
  &\geq 2 \cdot 2^k \cdot n! = 2^{k+1} \cdot n!,
  \end{align*}
  where the last inequality uses that $n+k+1 \geq 2$ when $n > 0$.
\end{proof}
  

\section{The sequence $a_n = 1/n!$}


\begin{definition}
Set $a_n = 1/n!$. This defines a function $a\colon \N \to \R, n \mapsto a_n$.
\end{definition}


\begin{theorem}[{\tt a\_bound}]\label{thm:a-bound}
For all $n \geq 1$ and $k \in \mathbb{N}$, we have
\[
a_{n+k}  \leq \left(\frac{1}{2}\right)^k \cdot a_n.
\]
\end{theorem}

\begin{proof}
Follows from Theorem~\ref{thm:fac-bound}, using the fact that both sides in the
inequality in Theorem~\ref{thm:fac-bound} are positive. 
\end{proof}

\section{Bounding the geometric series $g_n$}

Define
\[
  g_n = \sum_{i=0}^{n-1} (\frac{1}{2})^i = 1 + \cdots + (\frac{1}{2})^{n-1}
\]
Note that there are $n$ terms in the sum. By convention, $g_0 = 0$ (empty sum).

\begin{lemma}[{\tt g\_formula}] $g_n = 2 - 2 \cdot (1/2)^n$. \qed
\end{lemma}

\begin{corollary}[{\tt g\_lt\_2}] \label{cor:g-bound}
  $g_n < 2$. \qed
\end{corollary}

\section{The partial sums $s_n$}

Define the partial sums
\[
s_n = \sum_{i=0}^{n-1} a_i = \frac{1}{0!} + \frac{1}{1!} + \cdots + \frac{1}{(n-1)!} \in \R
\]
Note that there are $n$ terms in the sum. 


\begin{theorem}[{\tt s\_geometric\_bound}]\label{thm:s-below-geometric}
For all $m$ and all $n \geq 1$ and $k \in \mathbb{N}$, we have
\[
s_{n+k} \leq s_n + a_n \cdot g_k.
\]
\end{theorem}

\begin{proof}
The proof proceeds by induction on $k$. For $k = 0$, this is clear. Assume $s_{n+k} \leq s_n + a_n \cdot g_k$. Then
\begin{align*}
s_{n+k+1} &= s_{n+k} + a_{n+k} \\
&\leq s_n + a_n \cdot g_k + a_{n+k} \\
&\leq s_n + a_n \cdot g_k + \left(\frac{1}{2}\right)^k \cdot a_n \\
&= s_n + a_n \cdot \left(g_k + \left(\frac{1}{2}\right)^k\right) \\
&= s_n + a_n \cdot g_{k+1},
\end{align*}
where we used the bound $a_{n+k} \leq (1/2)^k \cdot a_n$ from Theorem~\ref{thm:a-bound}.
\end{proof}


It will be convenient to establish the following variant:

\begin{theorem}[{\tt s\_key\_bound}]\label{thm:s-bound-total}
For all $n \geq 1$ and all $m \geq 0$ we have
\[
  s_m < s_n + 2 \cdot a_n
\]
\end{theorem}

\begin{proof}
Follows by case distinction: if $m\geq n$ then $m =n+k$ for some $k$ and this follows from
Theorem~\ref{thm:s-below-geometric} and Corollary~\ref{cor:g-bound}.
\end{proof}

\section{Integrality and rationality}

A real number $x$ is called \emph{integral} if $x$ lies in $\Z$ and \emph{rational} if it lies
in $\Q$. 

\begin{theorem}[{\tt isInt\_fac\_mul\_s}]\label{thm:int-fac-mul-s}
If $m \geq n-1$, then $m! \cdot s_n$ is integral
\end{theorem}

\begin{proof}
Follows from the fact that $m! \cdot a_k$ is integral for $k \leq m$.
\end{proof}

\begin{theorem}[{\tt rationality\_criterion}] \label{thm:rationality-criterion} If $x$ is rational,
  then there exists an $n\geq 2$ such that $n! \cdot x$ is integral. 
\end{theorem}

\begin{proof}
For $n$ sufficiently large, the denominator of $x$ will divide $n!$.
\end{proof}

\section{The number $e = \lim_{n \to \infty} s_n$}

The number $e$ is defined as the limit of the sequence $(s_n)$:
\[
e = \lim_{n \to \infty} s_n = \sum_{n=0}^{\infty} \frac{1}{n!}.
\]

Since $(s_n)$ is an increasing sequence  and bounded
above (by Lemma~\ref{thm:s-bound-total}), the limit exists. 

\begin{lemma}[{\tt s\_lt\_e}, {\tt e\_le\_of\_s\_le}]\label{lemma:e-characterisation} The number $e$ satisfies:
    \begin{enumerate}
       \item For all $n$, we have $s_n < e$;
       \item If $c\in \R$ and $s_n \leq c$ for all $n$, then $e \leq c$.
    \end{enumerate}
 \qed
\end{lemma}

Note that $e$ is the \emph{only} real number satisfying this, so you can use
this lemma as a \emph{definition} of $e$. Combining the second point with Theorem~\ref{thm:s-bound-total} we obtain:

\begin{theorem}[{\tt key\_bound\_e}]\label{thm:e-bound}
  For all $n>0$ we have $e \leq s_n + 2 \cdot a_n$
\end{theorem}


\section{The tail $t_n$ of the series}

Define
\[
  t_n = e - s_n = \sum_{i\geq n} a_i = \frac{1}{n!} + \frac{1}{(n+1)!} + \cdots
\]

\begin{lemma}[{\tt t\_le\_twice\_a}]For $n\geq 1$ we have $t_n \leq 2 \cdot a_n$.
\end{lemma}

\begin{proof}
Follows from Theorem \label{thm:e-bound}.
\end{proof}

\begin{theorem}[{\tt fac\_mul\_t\_succ\_pos}, {\tt fac\_mul\_t\_succ\_lt\_one}]
  \label{thm:fac-mul-t-bound}
  For $n\geq 2$, 
  we have
  \[
     0 < (n!) \cdot t_{n+1} < 1
  \]
\end{theorem}

\begin{proof}
For the upper bound, we use:
  \begin{align*}
  n! \cdot t_{n+1} &\leq n! \cdot 2 \cdot a_{n+1} \\
  &= n! \cdot 2 \cdot \frac{1}{(n+1)!} \\
  &= \frac{2}{n+1} \leq \frac{2}{3} < 1,
  \end{align*}
  where we use $n \geq 2$.
\end{proof}



\begin{corollary}\label{cor:fac-mul-t-not-integral}
If $n\geq 2$, then $n! \cdot t_{n+1}$ is not integral.
\end{corollary}

\begin{proof}
Direct consequence of Theorem~\ref{thm:fac-mul-t-bound}: there is no integer strictly
between $0$ and $1$.
\end{proof}

\section{Irrationality of $e$}

\begin{theorem}[Irrationality of $e$]
The number $e$ is irrational.
\end{theorem}

\begin{proof}[Proof sketch]
Suppose for contradiction that $e$ is rational.  By theorem \label{thm:rationality-criterion}
there is an $n\geq 2$ such that $n! \cdot e$ is integral. By Theorem~\ref{thm:int-fac-mul-s}
also $n! \cdot s_{n+1}$ is integral. But then we have that also
\[
  n! t_{n+1} = n! \cdot e - n! \cdot s_{n+1}
\]
is integral, which contradicts Corollary~\ref{cor:fac-mul-t-not-integral}.
\end{proof}

\end{document}

