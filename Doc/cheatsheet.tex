\documentclass[10pt,landscape]{article}
\usepackage[utf8]{inputenc}
\usepackage[landscape,margin=0.5in]{geometry}
\usepackage{tabularx}
\usepackage{booktabs}
\usepackage{fancyvrb}
\usepackage{amsfonts}
\usepackage{multicol}

\DefineVerbatimEnvironment{code}{Verbatim}{fontsize=\small}

\title{Lean Tutoraat cheat sheet}
\author{}
\date{}

\begin{document}

\maketitle

\section*{Tactics}

\begin{table}[h!]
\centering
\begin{tabularx}{\textwidth}{l p{10cm} X}
\toprule
\textbf{\!Tactic} & \textbf{\!Usage} & \textbf{\!Example} \\
\midrule

\texttt{rfl} & 
Prove equalities that hold \emph{by definition}. &
\parbox[t]{8cm}{\ttfamily\small
example :\ 1 + 2 = 3 := by rfl
} \\

\midrule

\texttt{numbers} & 
Prove (in)equalities between purely numerical expressions. &
\parbox[t]{12cm}{\ttfamily\small
example :\ 5 \^{} 3 < 2 \^{} 7 := by numbers
} \\

\midrule

\texttt{algebra} & 
Prove algebraic identities. &
\parbox[t]{12cm}{\ttfamily\small
example (x y :\ $\mathbb{R}$) :\ (x + y) * (x - y) = x \^{} 2 - y \^{} 2 := by algebra
} \\ 

\midrule

\texttt{rewrite [h]} & 
If hypothesis \texttt{h} is of the form \texttt{a = b}, replace \texttt{a} with \texttt{b} in the goal. &
\parbox[t]{12cm}{\ttfamily\small
example (x :\ $\mathbb{Q}$) (h :\ x = 2) :\ x \^{} 2 = 4 := by\\
\hspace*{1em}rewrite [h];\ numbers
} \\ 

\midrule

\texttt{rewrite [$\leftarrow$h]} & 
If hypothesis \texttt{h} is of the form \texttt{a = b}, replace \texttt{b} with \texttt{a} in the goal. &
\parbox[t]{12cm}{\ttfamily\small
example (x y :\ $\mathbb{Q}$) (h :\ x + 1 = y) :\ x = y - 1 := by\\
\hspace*{1em}rewrite [$\leftarrow$h];\ algebra
} \\ 

\midrule

\texttt{positivity} & 
Prove goals of the form \texttt{a > 0} or \texttt{a $\geq$ 0}. &
\parbox[t]{12cm}{\ttfamily\small
example (x :\ $\mathbb{R}$) :\ x \^{} 2 $\geq$ 0 := by positivity
} \\ 

\midrule

\texttt{extra} & 
Prove inequalities of the form \texttt{a + e > a} or \texttt{a + e $\geq$ a}. &
\parbox[t]{12cm}{\ttfamily\small
example (a b :\ $\mathbb{Z}$) (h :\ a $\geq$ 0) :\ a + b $\geq$ b := by extra
} \\ 

\midrule

\texttt{rel [h]} & 
Use inequality \texttt{h} to prove a directly related inequality. &
\parbox[t]{12cm}{\ttfamily\small
example (a b :\ $\mathbb{R}$) (h :\ a $\geq$ b) :\ a + 1 $\geq$ b + 1 := by\\
\hspace*{1em}rel [h]
} \\ 

\midrule

\texttt{linarith} & 
A powerful tactic to automatically prove linear inequalities from linear (in)equalities 
in the hypotheses. &
\parbox[t]{12cm}{\ttfamily\small
example (a b :\ $\mathbb{R}$) (h1 :\ a > 2) (h2 :\ a + b < 3) :\ b < 1 := by\\
\hspace*{1em}linarith
} \\ 


\midrule

\texttt{calc} & 
Chain (in)equalities together to prove an (in)equality. &
\parbox[t]{12cm}{\ttfamily\small
example (x y :\ $\mathbb{R}$) :\ x \^{} 2 + y \^{} 2 - 2 * x * y $\geq$ 0 := by\\
\hspace*{1em}calc\\
\hspace*{2em}x \^{} 2 + y \^{} 2 - 2 * x * y = (x - y) \^{} 2 := by algebra\\
\hspace*{15em}\_ $\geq$ 0\hspace*{4.4em}:= by positivity
} \\ 

\midrule

\texttt{use} & 
Start proving an existential statement ($\exists$) by providing the value that
you will use. &
\parbox[t]{12cm}{\ttfamily\small
example :\ $\exists$ n :\ $\mathbb{N}$, 81 * n = 2025 := by\\
\hspace*{1em}use 25\\
\hspace*{1em}numbers
} \\ 

\midrule

\texttt{intro} & 
Start proving a universal statement ($\forall$) by introducing an arbitrary
variable. &
\parbox[t]{12cm}{\ttfamily\small
example :\ $\forall$ x :\ $\mathbb{R}$, x \^{} 2 $\geq$ 0 := by\\
\hspace*{1em}intro x\\
\hspace*{1em}positivity
} \\ 

\bottomrule
\end{tabularx}
\end{table}

\newpage

\section*{Typing mathematical symbols}

\begin{tabular}{cll}
\toprule
\textbf{Symbol} & \textbf{Type} & \textbf{Description} \\
\midrule
$\mathbb{N}$ & \texttt{\textbackslash N} & natural numbers \\
$\mathbb{Z}$ & \texttt{\textbackslash Z} & integers \\
$\mathbb{Q}$ & \texttt{\textbackslash Q} & rational numbers \\
$\mathbb{R}$ & \texttt{\textbackslash R} & real numbers \\
$\leq$ & \texttt{\textbackslash le} & less than or equal to \\
$\geq$ & \texttt{\textbackslash ge} & greater than or equal to \\
$\pi$ & \texttt{\textbackslash pi} & pi \\
$\leftarrow$ & \texttt{\textbackslash l} & used in \texttt{rewrite [$\leftarrow$h]} \\
$a^{-1}$ & \texttt{\textbackslash inv} & inverse \\
$\forall$ & \texttt{\textbackslash forall} & for all \\
$\exists$ & \texttt{\textbackslash exists} & exists \\
$\cdot$ & \texttt{\textbackslash .} & bullet (used for induction cases) \\
\bottomrule
\end{tabular}

\end{document}
